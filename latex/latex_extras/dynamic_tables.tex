%This helps with custom automatic tables. It uses latex  functions
%http://www.jwe.cc/2012/03/stata-latex-tables-estout/
%http://repec.org/bocode/e/estout/estout.html

% *****************************************************************
% Estout related things
% *****************************************************************

% Added due to changes in TexLive 2022 breaking code
%https://www.overleaf.com/blog/tex-live-2022-now-available
%https://tex.stackexchange.com/questions/567985/problems-with-inputtable-tex-hline-after-2020-fall-latex-release
\makeatletter
%%primitive input in tabular
\AddToHook{env/tabular/begin}{\let\estinput=\@@input}
\makeatother
%\let\estinput=\@@input % define a new input command so that we can still flatten the document

\newcommand{\estwide}[3]{
    \vspace{.25ex}{
      \textsymbols% Note the added command here
      \begin{tabular*}
      {\textwidth}{@{\hskip\tabcolsep\extracolsep\fill}l*{#2}{#3}}
      \toprule
      \estinput #1 
      \bottomrule
      \addlinespace[.75ex]
      \end{tabular*}
      }
    } 

\newcommand{\estauto}[3]{
    \vspace{.25ex}{
      \textsymbols% Note the added command here
      \begin{tabular}{l*{#2}{#3}}
      \toprule
      \estinput #1 
      \bottomrule
      \addlinespace[.75ex]
      \end{tabular}
      }
    }

% Allow line breaks with \\ in specialcells
\newcommand{\specialcell}[2][c]{%
    \begin{tabular}[#1]{@{}c@{}}#2\end{tabular}
}
\DeclareUnicodeCharacter{00A0}{ }
% *****************************************************************
% Custom subcaptions
% *****************************************************************
% Note/Source/Text after Tables
% The new approach using threeparttables to generate notes that are the exact width of the table.
\newcommand{\Figtext}[1]{%
  \begin{tablenotes}[para,flushleft]
  %\hspace{6pt}
  %\hangindent=1.75em
  #1
  \end{tablenotes}
  }
\newcommand{\Fignote}[1]{\Figtext{\text{Notes:~}~#1}}
\newcommand{\Figsource}[1]{\Figtext{\text{Source:~}~#1}}
\newcommand{\Starnote}{\Figtext{Point estimates marked ***, **, and * are statistically significant at the 1, 5, and 10 percent levels, respectively.}}% Add significance note with \starnote
% If you are using hyper-ref (recommended), this command must go after all 
% other package inclusions (from the hyperref package documentation).
% The purpose of hyperref is to make the PDF created extensively
% cross-referenced.

\usepackage{ifthen}
\newboolean{stars_on}
\setboolean{stars_on}{false} % or {false}
\ifthenelse{\boolean{stars_on}}
{%
  % Text to include when stars_on is true
  \newcommand{\sym}[1]{\rlap{#1}}
  \newcommand{\nlsym}[1]{\rlap{\append{}{*}{#1}}}
  \newcommand{\pvaluenote}{* p $<$ 0.1, ** p $<$ 0.05, *** p $<$ 0.01. }
}
{%
  % Text to include when stars_on is false
  \newcommand{\sym}[1]{\rlap{~}}
  \newcommand{\nlsym}[1]{\rlap{\append{}{~}{#1}}}
  \newcommand{\pvaluenote}{}

}
 %Allows for fancy stars in tables that also works in beamer
% %Allows for fancy stars in tables that also works in beamer

% Create a function that works for the non-linear symbols
%\newcommand{\nlsym}[1]{\rlap{\append{}{*}{#1}}} %Allows for fancy stars in tables that also works in beamer
 %Allows for fancy stars in tables that also works in beamer

% Character substitution that prints brackets and the minus symbol in text mode. Thanks to David Carlisle
\def\yyy{%
  \bgroup\uccode`\~\expandafter`\string-%
  \uppercase{\egroup\edef~{\noexpand\text{\llap{\textendash}\relax}}}%
  \mathcode\expandafter`\string-"8000 }

\def\xxxl#1{%
\bgroup\uccode`\~\expandafter`\string#1%
\uppercase{\egroup\edef~{\noexpand\text{\noexpand\llap{\string#1}}}}%
\mathcode\expandafter`\string#1"8000 }

\def\xxxr#1{%
\bgroup\uccode`\~\expandafter`\string#1%
\uppercase{\egroup\edef~{\noexpand\text{\noexpand\rlap{\string#1}}}}%
\mathcode\expandafter`\string#1"8000 }

\def\textsymbols{\xxxl[\xxxr]\xxxl(\xxxr)\yyy}


%Protects against odd minus sign errors within the tables
\catcode`_11
\protected\def \c__siunitx_minus_tl {$-$}
\catcode`_ 8 

%-----------Tables-------------%
% This Section is needed to make the tables work with amsmath. But because of the \( being open and having no match, Sublime Text thinks that everything after it (all text) is in math mode and hence changes the text color. This annoys me so the only ``solution'' is to just place this preamble here. It seems to work still. 
\makeatletter 
\edef\originalbmathcode{%
    \noexpand\mathchardef\noexpand\@tempa\the\mathcode`\(\relax} %\)

\def\resetMathstrut@{%
  \setbox\z@\hbox{%
    \originalbmathcode

    \def\@tempb##1"##2##3{\the\textfont"##3\char"}%
    \expandafter\@tempb\meaning\@tempa \relax
  }%
  \ht\Mathstrutbox@\ht\z@ \dp\Mathstrutbox@\dp\z@
}
\makeatother %./tables.tex:25: LaTeX Error: Bad math environment delimiter. [\)]
\usepackage{xparse}

\ExplSyntaxOn
\DeclareExpandableDocumentCommand{\append}{mmm}
 {
  #1\prg_replicate:nn{#3}{#2}
 }
\ExplSyntaxOff

\doublespacing